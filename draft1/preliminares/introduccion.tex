% !TeX root = ../tfg.tex
% !TeX encoding = utf8
%
%*******************************************************
% Introducción
%*******************************************************

% \manualmark
% \markboth{\textsc{Introducción}}{\textsc{Introducción}} 

\chapter{Introducción}

El problema presentado es el siguiente: sea $U$ un conjunto cualquiera, no vacío, y sea $F=\{U_1, U_2, ..., U_m; U_i\neq \emptyset, \forall i \in \{1,...,m\}\}$ un conjunto de un número finito $m\in \mathbb{N}$ de subconjuntos de $U$. Sea $G\subseteq U$ un subconjunto cualquiera. Nuestro principal objetivo es  aproximar lo mejor posible el conjunto $G$, por elementos de $F$. Es decir, queremos encontrar los $U_i$ que maximizan la parte de $G$ que queda cubierta, minimizando a su vez la parte de los $U_i$ escogidos que quedan fuera de $G$ y minimizando también la intersección o el solapamiento entre los subconjuntos escogidos, para evitar repetir elementos. Lo ideal sería encontrar ciertos $U_i$ tal que $\bigcup_i U_i = G$ y $\bigcap_i U_i = \emptyset$, donde $G$ queda totalmente cubierto por subconjuntos disjuntos que no contienen elementos que no pertenecen a $G$. Como esto no siempre es posible, utilizaremos diferentes medidas como el Dice Score, Jaccard Index, precision o el recall para escoger los conjuntos que mejor aproximen $G$ según diferentes criterios. Para ello, en el ámbito matemático profundizaremos en teoría de conjuntos, álgebra de conjuntos y teoría de anillos, entendiendo la naturaleza del problema presentado, y estudiando qué ocurre al aplicar ciertas operaciones entre subconjuntos, como intersección, unión, complemento... Nos planteamos también cuestiones paralelas cómo: ¿qué ocurriría si $F$ no fuese un recubrimiento de $U$? ¿Podríamos incluir $U$?, a las que también daremos respuesta. Sin pérdida de generalidad, hemos supuesto que $F$ no contiene ningún elemento vacío, pues no aportaría información a nuestro problema. En principio, $F$ no tiene ninguna otra propiedad específica. Enlazando con la parte informática, implementaremos varios algoritmos diferentes para la elección de estos subconjuntos que recubriran $G$, basándonos en alguna medida de las mencionadas. Impondremos además un parámetro $k$, calculado empíricamente, que representará el número máximo de operaciones entre los subconjuntos, o el número máximo de subconjuntos a tener en cuenta, para alinearnos con las limitaciones que tenemos en capacidad de cómputo. Este problema surge precisamente como propuesta de mis tutores, a raíz de investigaciones en las que $G$ contiene millones de elementos. 

De acuerdo con la comisión de grado, el TFG debe incluir una introducción en la que se describan claramente los objetivos previstos inicialmente en la propuesta de TFG, indicando si han sido o no alcanzados, los antecedentes importantes para el desarrollo, los resultados obtenidos, en su caso y las principales fuentes consultadas.

Ver archivo \texttt{preliminares/introduccion.tex}

\endinput
